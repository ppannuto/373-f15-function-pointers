\documentclass{article}
\usepackage[top=.75in, bottom=.75in, left=.75in,right=.75in]{geometry}
\usepackage{fancyhdr}
\usepackage[table]{xcolor}
\usepackage[colorlinks,urlcolor={blue},pdfstartview=FitH]{hyperref}

\makeatletter
\renewcommand\section{\@startsection {section}{1}{\z@}%
  {-3.5ex \@plus -1ex \@minus -.2ex}%
  {2.3ex \@plus.2ex}%
  {\normalfont}%
}
\renewcommand\subsection{\@startsection {subsection}{1}{\z@}%
  {-3.5ex \@plus -1ex \@minus -.2ex}%
  {2.3ex \@plus.2ex}%
  {\normalfont}%
}
\makeatother

\begin{document}

\pagestyle{fancyplain}

\title{\vspace{-2em}\textbf{Homework 4}}
\author{Pointers and the Funny Things You Can Do With Them}
\date{}
\maketitle

\begin{Form}
% Don't use hyperref's text label
\def\LayoutTextField#1#2{#2}

\begin{flushright}
\begin{tabular}{r l}
  Name     & \TextField[width=5cm,charsize=10pt,color=black,bordercolor=,backgroundcolor=.95 .95 .95]{name} \\
           & \\[-1em]
  uniqname & \TextField[width=5cm,charsize=10pt,color=black,bordercolor=,backgroundcolor=.95 .95 .95]{uniqname} \\
\end{tabular}
\end{flushright}

\section{The code measures the ``effort'' of each sorting algorithm.\\What is
effort in this case (what is it actually measuring)?\\How is it measured?\\
\textbf{[5 points]}
}

\TextField[width=\hsize,height=80pt,multiline=true,charsize=10pt,color=black,bordercolor=,backgroundcolor=.95 .95 .95]{q1}

\subsection{\emph{Extra Credit (optional):}\\
On modern CPUs, this may not be a good way of measuring effort.\\
Explain one way it can be inaccurate and suggest a better measurement method.\\
\textbf{[up to 2 points]}
}

\TextField[width=\hsize,height=80pt,multiline=true,charsize=10pt,color=black,bordercolor=,backgroundcolor=.95 .95 .95]{q1-ec}

\section{The end of \texttt{main} is printing something out.\\
What is the data it's printing?\\What do those bytes do when the program is run?\\
\textbf{[5 points]}
}

Copy the output from main here:
\TextField[width=12.5cm,charsize=10pt,color=black,bordercolor=,backgroundcolor=.95 .95 .95]{q2-output}

\medskip

\noindent
\TextField[width=\hsize,height=80pt,multiline=true,charsize=10pt,color=black,bordercolor=,backgroundcolor=.95 .95 .95]{q2}

\end{Form}

\bigskip
\hrule
\bigskip
\begin{flushright}
\begin{tabular}{r r}
  \textbf{Score from the programming portion} & \_\_\_ / 20 \\
  \textbf{Score from the written portion}     & \_\_\_ / 10 \\
\end{tabular}
\end{flushright}
\bigskip
\hrule

\end{document}

