\documentclass{article}
%\usepackage{fullpage}
%\usepackage[top=1in, bottom=1in, left=1cm,right=1cm]{geometry}
\usepackage[top=.75in, bottom=.75in, left=.75in,right=.75in]{geometry}
\usepackage{fancyhdr}
\usepackage{color}
\usepackage[colorlinks,urlcolor={blue}]{hyperref}

\begin{document}

%\fancyfoot[L]{Revision 1.0 -- 12/07/2011}
\pagestyle{fancyplain}


\title{\textbf{Homework 4\\Pointers and the Funny Things You Can Do With Them}}
\author{Assigned: Monday, September 28, 1:30PM}
\date{\textbf{\color{red}{Due: Monday, October 5, 1:30PM (Hard Deadline)}}}
\maketitle

The purpose of this assignment is to act as a refresher on how to work with
pointers, especially function pointers, in C. Additionally, it refreshes the
new concept of weak references.


\section*{Grading}

\emph{Part I: Software}

\begin{itemize}
  \item 20 points -- Everything works.
  \item 10 points -- You put effort in but it doesn't quite work.
  \item 0 points -- Little to no effort / nothing works / nothing submitted.
\end{itemize}

\noindent
\emph{Part II: Follow-up Questions}

\begin{itemize}
  \item 10 points --
    \href{http://web.eecs.umich.edu/~prabal/teaching/eecs373/homeworks/373-F15-HW4-followup.pdf}{Follow-up questions}
\end{itemize}

\noindent
\textit{Honor Code Reminder:} In the interest of making this easy for you to
implement, test, and submit this is probably one of the easiest assignments to
cheat on ever. Please don't. Resist the temptation to ``glance'' at another
solution for help because you got stuck. There just isn't enough code that a
``glance'' will do anything but give you the solution. This assignment is
deliberately worth very few points; if you didn't do it, just don't submit
anything [why risk cheating?]. The point of this assignment is to make your
life easier later this semester, don't cheat yourself.


\section*{Assignment}

Grab a copy of the code from
\url{http://github.com/eecs373-f15/373-f15-function-pointers}.
I recommend forking this repository to do your work.

\medskip
\noindent
First things first, type \texttt{make main \&\& ./main} and look at the output.
Look over \texttt{main.c} and \texttt{sort.c} to try to understand what the
current code is doing. Specifically, how does \texttt{main} call the different
sort functions in \texttt{sort.c}? How does \texttt{main} know how many sort
functions there are? What is the purpose of the \texttt{compare} function?

\bigskip
\noindent
\textbf{Task 1:} We will add a third sort algorithm. Instead of implementing
it ourselves, we will use the built-in \texttt{qsort} from the C standard
library. For details on \texttt{qsort}, type
\texttt{\href{http://linux.die.net/man/3/qsort}{man qsort}}. The
\texttt{qsort} type signature doesn't quite match our \texttt{sorting\_fn}
type signature, you will need to write a wrapper function.

\smallskip
\noindent
Do your work for Task 1 in the \texttt{sort.c} file. You may only edit
\texttt{sort.c} for this task.

\smallskip
\noindent
When you are done with this task, type \texttt{make check\_main} to check your work.

\bigskip
\noindent
\textbf{Task 2:} Next we will modify the sort functions to reverse the order
they are sorting in. We will do this \emph{without} modifying \texttt{sort.c}.
Type \texttt{make reverse \&\& ./reverse}. Currently this will behave the same
as \texttt{main}. The difference between \texttt{main} and \texttt{reverse} is
that \texttt{reverse} also links in \texttt{reverse\_sort.c}.
Add a function to \texttt{reverse\_sort.c} so that numbers are now sorted in
descending order.
You may find it useful to consult Lab~4 for a refresher on weak references.

\smallskip
\noindent
Do your work for Task 2 in the \texttt{sort.c} file. You may only edit
\texttt{reverse\_sort.c} for this task.

\smallskip
\noindent
When you are done with this task, type \texttt{make check} to check your work.

\section*{Notes}
I have tested this on my (Linux) desktop, the CAEN Linux machines, and my
personal mac. If you have any issues building / running the initial code,
please file an issue on GitHub (the ``Issues'' button at the right side of the
page).


\section*{Submission}

\emph{Part I: Software}

\medskip
\noindent
Your \texttt{sort.c} and \texttt{reverse\_sort.c} implementations will be
automatically graded. To submit, go to
\url{https://docs.google.com/spreadsheets/d/1AGWtXCvt7iAtThuRlz3XovYnD2dfjzQA6idmVA6sM80/edit?usp=sharing}
and add your uniqname and a link to download your \texttt{sort.c} and
\texttt{reverse\_sort.c}.

\medskip
\noindent
A script will automatically download, compile, and grade your \texttt{sort.c}
and \texttt{reverse\_sort.c} implementations. This script will run
automatically at 1:35PM on Monday, October 5th (to account for clock
differences, don't rely on this).
\textbf{\color{red}No late submissions will be accepted.}
A test run of the script will run at 1:35PM on Sunday, October 4th. In both
cases, the script will immediately email the results.


\bigskip
\noindent
\emph{Part II: Follow-up Questions}

\medskip
\noindent
We will be trying Gradescope for this part of the assignment. Download the follow-up questsions
\href{http://web.eecs.umich.edu/~prabal/teaching/eecs373/homeworks/373-F15-HW4-followup.pdf}{here}.
This is an editable PDF, so you should be able to simply type in your
responses (be sure to save a copy with your responses!). If that doesn't work,
you can simply print and scan a copy of the assignment.
\textbf{TODO: Add gradescope directions once I have an account set up; waiting
on review from them right now.}


\end{document}
